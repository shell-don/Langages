\documentclass[conference]{IEEEtran}

\usepackage[utf8]{inputenc}
\usepackage{graphicx}
\usepackage{caption}
\usepackage[hidelinks]{hyperref}

\title{Apprentissage des langues informatiques}
\author{
    \IEEEauthorblockN{Pigassou Mathis\IEEEauthorrefmark{1}} 
    \IEEEauthorblockA{UFR Informatique, Université Toulouse Capitole, France\\
    Email: mathis.pigassou@ut-capitole.fr}
}

\begin{document}

\maketitle  

\begin{abstract}  
Livres ou cours magistraux, la pratique est sûrement la méthode la plus efficace pour l'apprentissage des langages de programmation.
\end{abstract}
\begin{IEEEkeywords} 
Python, C, Shell, LaTeX 
\end{IEEEkeywords}


\section{Introduction} 
Souvent auréolé d'une aura mystique, le programmeur informatique est parfois idéalisé, car, touché par la grâce de l'abstraction, il sait faire ce que l'on peut croire impossible : maîtriser la machine. \newline
Seulement voilà, un programmeur n'est qu'un traducteur d'algorithmes (suite d'opérations résolvant un problème). S'il n'est pas déjà créé, alors il le fera. \newline
Chaque langue a ses propres règles. Certaines ont leurs propres expressions, d'autres ont une structure particulière. Un bon programmeur est, avant toute chose, un bon lecteur. 

 
\section{Méthodologie}
En effet, l'apprentissage d'un langage passe d'abord par des lectures aussi nombreuses que variées. Le \emph{plaisir} de lire est essentiel. La \emph{curiosité} qui en coule est la sève de l'envie nourrissant l'arbre de la connaissance \ref{fig:ARBRE}. La \emph{pratique} viendra alors frapper du sceau de l'immortalité cette connaissance et soufflera sur le feu de ce cercle vertueux.   

\section{Résultats}
Preuve d'un théorème, bitcoin simplifié, anomalie statistique dans les élections russes, graphes, MasterMind, création d'images, calculatrice polonaise, résolution de systèmes, génération de documents, fractales et L-système... \newline 
Dans la forme, ce dossier contient quelques supports théoriques puis une partie pratique (TP) contenant les sujets ainsi que les programmes associés \footnote{Les programmes ne sont pas exhaustifs, sûrement optimisables.}. Les travaux pratiques participent aussi à l'étude de nouvelles branches plus générales, non liées à l'informatique \footnote{Par exemple pour dessiner un arbre avec un L-système, il faut d'abord comprendre ce qu'est un L-système. (issue de la biologie)}. \newline 
Dans le fonds, il s'agit de partager mon expérience. Ainsi, une comparaison peut être pertinente dans la mesure où elle peut permettre une autocritique constructive.


\begin{figure}[!t]
    \centering
    \includegraphics[width=0.374\textwidth]{ARBRE.png}
    \captionsetup{labelformat=empty}
    \caption{ARBRE}
    \label{fig:ARBRE} 
\end{figure}


\section{Conclusion}
La programmation est un moyen, moyen de pousser et de faire pousser ses connaissances. Parfois de planter le germe d'autres.


\end{document}
